\chapter{Conclusion}\label{ch:conclusion}

This dissertation presented a search for a nonresonant excess of high-mass diphoton events over the Standard Model background prediction produced from proton-proton collisions at a center-of-mass energy of 13\TeV. The data sample used corresponded to an integrated luminosity of 35.9\fbinv, collected by the CMS detector at the Large Hadron Collider, and constituted the full 2016 dataset. Such an excess could be a signature of large extra dimensions, as described by the scenario of Arkani-Hamed, Dimopoulos, and Dvali. \correction{This model proposes a solution to the Standard Model hierarchy problem by modifying the fundamental Planck scale \MD to be on the order of the electroweak scale.}

The large-extra dimensional signal was modeled using the \SHERPA Monte Carlo event generator, taking into account the non-negligible interference effects between the signal and background. \correction{This model is parameterized by the the number of extra dimensions \nED and the string mass scale \Ms, which is directly proportional to \MD.} Various conventions for constraining the model were considered, spanning a parameter space ranging between $3 < \Ms < 11\TeV$ for $\nED=2$-7. The diphoton invariant mass region above $500\GeV$ was used for this signal search.

A next-to-next-to-leading order Monte Carlo calculation was used to predict the dominant, irreducible Standard Model diphoton background using the \SHERPA Monte Carlo event generator combined with the \MCFM Monte Carlo parton-level calculator. To account for additional higher order terms absent in this calculation, the normalization of predicted Standard Model background was allowed to float, under constraint from the data. A technique using control samples in data was used to measure the subdominant, reducible fake background, primarily arising when one or two jets fragment in such a way as to mimic a photon signature, causing them to be misreconstructed as diphoton events in the CMS detector. Templates representing real and fake photon distributions were fit to data to yield a measurement of the total fake contribution. 

The data are found to be in agreement with the total background predicted from Standard Model sources, consistent with the background-only hypothesis. No evidence for new physics is observed. For the large extra-dimensional model of Arkani-Hamed, Dimopoulos, and Dvali, exclusion limits at 95\% confidence level are set in the range $5.6 < \Ms < 9.7\TeV$, depending on the specific model convention. \correction{This range starts to surpass the range of the electroweak scale.} This result extends the current best lower limits on \Ms from the diphoton channel, as presented in Ref.~\cite{Aaboud:2017yyg}.

This is the first search by the CMS Collaboration for a nonresonant excess in the high-mass diphoton channel performed during the LHC Run~2 data taking period. The previous CMS result used 2.2\fbinv of Run~1 data collected in 2011 at a center-of-mass energy of 7\TeV~\cite{Chatrchyan:2011fq}. The techniques used in Run~1 formed the basis of this search and were improved, in particular, by utilizing a next-to-next-to-leading order Monte Carlo calculation of the prompt Standard Model background and incorporating data from the CMS detector endcaps. Extensions of this result are expected by a planned combination with the CMS dilepton channel, as well as a future combination with the full 2017 and 2018 diphoton datasets, which would constitute the CMS Run~2 legacy result in the high-mass diphoton channel. 

The results of this dissertation, along with an analogous search for a resonant excess in the high-mass diphoton channel, have been published in Ref.~\cite{Sirunyan:2018wnk}. The existence of extra spatial dimensions in the universe has not been ruled out, but their hope of solving the Standard Model hierarchy problem has been narrowed by this effort.